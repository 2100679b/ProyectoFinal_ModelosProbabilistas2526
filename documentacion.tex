\documentclass[12pt, a4paper]{book} % Usamos 'book' para un documento extenso con capítulos

% ==========================================================
% PAQUETES ESENCIALES
% ==========================================================
\usepackage[utf8]{inputenc}
\usepackage[T1]{fontenc}
\usepackage[spanish]{babel}    % Soporte para español
\usepackage{amsmath}           % Matemáticas avanzadas (alineación, matrices)
\usepackage{amssymb}           % Símbolos matemáticos (conjuntos, distribuciones)
\usepackage{graphicx}          % Para incluir figuras y diagramas
\usepackage{hyperref}          % Para enlaces interactivos en el PDF
\usepackage{listings}          % Para incluir código fuente (PHP)
\usepackage{geometry}          % Para definir márgenes
\usepackage{float}             % Para control de posicionamiento de figuras [H]
\usepackage{verbatim}          % Para entorno verbatim (mostrar texto como está)

% Configuración de Márgenes
\geometry{
 a4paper,
 total={170mm,257mm},
 left=25mm,
 right=20mm,
 top=25mm,
 bottom=20mm,
}

% ==========================================================
% CONFIGURACIÓN DEL CÓDIGO (LISTINGS)
% ==========================================================
\lstset{
    language=PHP,                 % Lenguaje a resaltar
    basicstyle=\small\ttfamily,   % Estilo de la fuente
    numbers=left,                 % Números de línea a la izquierda
    numberstyle=\tiny\color{gray},% Estilo de los números de línea
    stepnumber=1,                 % Mostrar cada línea
    showspaces=false,             % No mostrar espacios
    showstringspaces=false,       % No mostrar espacios en strings
    showtabs=false,               % No mostrar tabs
    frame=single,                 % Marco alrededor del código
    framesep=5pt,                 % Espacio entre el marco y el texto
    rulecolor=\color{black!70},   % Color del marco
    tabsize=4,                    % Tamaño de la tabulación
    captionpos=b,                 % Posición del título (abajo)
    breaklines=true,              % Romper líneas largas
    keywordstyle=\color{blue}\bfseries, % Estilo de las palabras clave
    stringstyle=\color{red},      % Estilo de los strings
    commentstyle=\color{green!60!black}\itshape % Estilo de los comentarios
}

% ==========================================================
% METADATOS DEL DOCUMENTO
% ==========================================================
\title{\textbf{Proyecto Final: Modelos Probabilísticos y Aplicaciones en PHP}}
\author{Nombre del Alumno / Equipo}
\date{Diciembre de 2025} % Fecha Actual

\begin{document}

\frontmatter % Para la portada y tabla de contenido (páginas en números romanos)
\maketitle
\tableofcontents

\mainmatter % Para el cuerpo principal (páginas en números arábigos)

% ==========================================================
% CAPÍTULO 1: MANUAL DE USUARIO (Requisito 1)
% ==========================================================
\chapter{Manual de Usuario - Instalación y Entorno}
\label{ch:manual}

Este capítulo es la guía esencial para la configuración y ejecución del proyecto de Modelos Probabilísticos. Si ya tiene \textbf{PHP} instalado (versión 7.4 o superior), puede avanzar a la Sección \ref{sec:clonacion}.

\section{Requisitos y Verificación de PHP}

El proyecto está desarrollado en \textbf{PHP puro} y se ejecuta desde la terminal.

\subsection{Verificación de PHP}
Abra su terminal (\textit{Símbolo del Sistema}, \textit{PowerShell}, o \textit{Terminal}) y ejecute:
\begin{verbatim}
php -v
\end{verbatim}

\begin{itemize}
    \item Si el comando retorna la versión (\textit{ej. PHP 8.2.10...}), la instalación es correcta.
    \item Si recibe un error, debe instalar PHP.
\end{itemize}

\section{Descarga del Código Fuente (Clonar desde GitHub)}
\label{sec:clonacion}

Utilizaremos \textbf{Git} para obtener el código fuente desde el repositorio.

\subsection{Verificación de Git}
Verifique la instalación de Git ejecutando:
\begin{verbatim}
git --version
\end{verbatim}

\subsection{Clonar el Repositorio}
Navegue a la ubicación deseada y use el comando \texttt{git clone} con la \texttt{URL} de su proyecto:
\begin{verbatim}
git clone [URL_DEL_REPOSITORIO] ProyectoFinal_ModelosProbabilistas2526
cd ProyectoFinal_ModelosProbabilistas2526
\end{verbatim}

La estructura de carpetas incluye los módulos principales en \texttt{modules/}:
\begin{itemize}
    \item \texttt{modules/bayesian} (Modelos Bayesianos)
    \item \texttt{modules/markov} (Cadenas de Markov)
    \item \texttt{modules/hmm} (Modelos Ocultos de Markov)
\end{itemize}

% ==========================================================
% CAPÍTULO 2: EXPLICACIÓN DE ALGORITMOS (Requisito 2)
% ==========================================================
\chapter{Explicación de los Algoritmos Implementados}
\label{ch:algoritmos}

Este capítulo detalla la formulación matemática y la implementación de los modelos probabilísticos utilizados en el proyecto.

\section{Modelos Bayesianos}

\subsection{El Teorema de Bayes}
El principio fundamental de nuestro módulo \texttt{bayesian} es el Teorema de Bayes para la inferencia probabilística:
\begin{equation}
P(\theta | \mathbf{X}) = \frac{P(\mathbf{X} | \theta) P(\theta)}{P(\mathbf{X})}
\label{eq:bayes}
\end{equation}
Donde $P(\theta | \mathbf{X})$ es la \textbf{posterior}, $P(\mathbf{X} | \theta)$ es la \textbf{verosimilitud}, $P(\theta)$ es la \textbf{prior} y $P(\mathbf{X})$ es la evidencia.

\section{Cadenas de Markov y Matrices de Transición}

El comportamiento del sistema es modelado por una Matriz de Probabilidad de Transición $A$, donde $a_{ij}$ representa la probabilidad de transitar del estado $i$ al estado $j$.
\begin{equation}
A = 
\begin{pmatrix}
p_{11} & p_{12} & \cdots & p_{1N} \\
p_{21} & p_{22} & \cdots & p_{2N} \\
\vdots & \vdots & \ddots & \vdots \\
p_{N1} & p_{N2} & \cdots & p_{NN}
\end{pmatrix}
\end{equation}

\section{Modelos Ocultos de Markov (HMM)}
\label{sec:hmm_algoritmos}

Para el módulo \texttt{hmm}, se implementaron los siguientes algoritmos:
\begin{itemize}
    \item \textbf{Forward-Backward (Evaluación)}: Para calcular la probabilidad de una secuencia de observaciones $P(\mathbf{O} | \lambda)$.
    \item \textbf{Viterbi (Descodificación)}: Para encontrar la secuencia de estados ocultos más probable $\mathbf{Q}^*$ que produjo la secuencia de observaciones $\mathbf{O}$.
\end{itemize}

% Se puede añadir un diagrama si tienes el archivo en 'assets/'
%\begin{figure}[H]
%    \centering
%    \includegraphics[width=0.8\textwidth]{assets/diagrama_hmm.png}
%    \caption{Representación gráfica de la estructura de un HMM.}
%    \label{fig:hmm_diagram}
%\end{figure}

% ==========================================================
% CAPÍTULO 3: DECISIONES DE DISEÑO (Requisito 3)
% ==========================================================
\chapter{Decisiones de Diseño y Arquitectura PHP}
\label{ch:diseno}

Este capítulo justifica las elecciones de diseño y la estructura del código implementada en \text{PHP}.

\section{Modularidad Basada en Modelos}
El proyecto utiliza una estructura de directorios modular (\texttt{bayesian}, \texttt{markov}, \texttt{hmm}) que se mapea directamente a las clases y a los problemas que resuelven.

\section{Representación de Estructuras de Datos}
\subsection{Matrices y Vectores}
Las matrices de probabilidad de transición y las distribuciones iniciales se representan en \text{PHP} usando \textbf{arrays asociativos} y \textbf{multidimensionales}. Esta elección optimiza la legibilidad y la indexación por nombre de estado, aunque podría tener implicaciones de rendimiento en comparación con extensiones optimizadas.

\section{Manejo de Errores y Validaciones}
Se ha implementado la validación estricta de entradas (ej. la suma de probabilidades en las matrices de transición debe ser igual a 1).

% ==========================================================
% CAPÍTULO 4: EJEMPLOS DE USO (Requisito 4)
% ==========================================================
\chapter{Ejemplos de Uso y Código Fuente}
\label{ch:ejemplos}

Los siguientes ejemplos demuestran cómo interactuar con los diferentes módulos del proyecto. Asumimos que ya se encuentra en el directorio raíz \texttt{ProyectoFinal\_ModelosProbabilistas2526}.

\section{Ejemplo 4.1: Clasificación con Naive Bayes}
\label{sec:ejemplo_bayes}
Este ejemplo ilustra la inicialización y el entrenamiento de un clasificador Naive Bayes simple.

\lstinputlisting[caption={Uso del Clasificador Bayesiano (\texttt{ejemplo\_bayes.php})}, label={lst:bayes}]{modules/bayesian/ejemplo_bayes.php}

\noindent Para ejecutar este ejemplo desde la terminal, use:
\begin{verbatim}
php modules/bayesian/ejemplo_bayes.php
\end{verbatim}

\section{Ejemplo 4.2: Predicción de Secuencias con HMM}
\label{sec:ejemplo_hmm}
Este script usa el algoritmo de Viterbi para inferir la secuencia de estados ocultos más probable dada una secuencia de observaciones.

\lstinputlisting[caption={Algoritmo de Viterbi (\texttt{ejemplo\_viterbi.php})}, label={lst:viterbi}]{modules/hmm/ejemplo_viterbi.php}

\noindent Para ejecutar este ejemplo, use:
\begin{verbatim}
php modules/hmm/ejemplo_viterbi.php
\end{verbatim}

\end{document}